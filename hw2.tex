\documentclass[11pt]{article}
\usepackage{fullpage}
\usepackage{amsmath,amssymb,amsfonts}
\renewcommand\vec[1]{\mathbf{#1}}
\usepackage{listings}
\usepackage[table]{xcolor}
\usepackage{framed,comment}

\newcommand\red[1]{\textcolor{red}{#1}}
\specialcomment{solution}{\bigskip\begin{leftbar}\par\noindent\textbf{Solution.} }{\end{leftbar} }
\begin{document}

\begin{tabular}{l}
\textbf{CSCI 5654-Fall16}: Assignment \#2 (Reading: Chapter 2 of Vanderbei's book). \\
\textbf{Due Date:} Friday,  September 16, 2016 (before class) \\
\textbf{In-class:} Assignment should be submitted on paper -- no emails. \\
  \textbf{Distance Students:} Assignment may be submitted on paper or
  by email. \\[10pt]

\textbf{Your Name:} \phantom{Andrew Candelaresi}\\
\hline
\\[10pt]
\end{tabular}

\textbf{P1. (10 points)} Consider the following feasible dictionary:

\[ \begin{array}{r| c c c c c c c}
w_1 & 8 & - 5 x_1 & - 4 x_2 & + x_5 & - 2 x_6 & + 2 w_3 & - w_ 4\\
w_2 & 2 & - x_1 &    &  - x_5 &  - x_6 & - w_3 & \\
x_3 & 2 &         & + x_2 & - x_5 & - x_6 & - w_3 & - 2 w_4 \\
x_4 & 3 & - x_1 & - x_2 & - x_5 & - x_6 & + 2 w_3 & + w_4 \\ 
w_5 & 4 & - x_1 & - 2 x_2 & - 4 x_5 &  &  & + w_4 \\
\hline
z &    11 &  - x_1 &  + 2 x_2 & + x_5 & + x_6 & + 3w_3 & - 2 w_4 \\
\end{array}\]


\noindent (A) Write down all possible entering variables. For each
entering variable, write down all possible leaving variables
corresponding to that entering variable. Express your answer as a
table such as the one sample shown below.


\begin{center}
\begin{tabular}{|c|c|}
\hline
Entering & Leaving \\
\hline
$x_2$ & $w_1\ \mbox{or}\ w_5$ \\
\hline
$x_5$ & $w_2\ \mbox{or}\ x_3\ \mbox{or}\ x_4\ \mbox{or}\ w_5$\\
\hline
$x_6$ & $w_1\ \mbox{or}\ w_2\ \mbox{or}\ x_3\ \mbox{or}\ w_4$\\
\hline
$w_3$ & $w_2\ \mbox{or}\ x_3$\\
\hline
\end{tabular}

\end{center}

\medskip

\noindent\textbf{(B)} Suppose $x_2$ were chosen to be the entering
variable and $w_5$ as the leaving variable. Write down the basic
variables, non basic variables and their corresponding solutions in
the  \emph{next} dictionary after pivoting. (It is preferable if you
answer this problem withing  attempting to actually work out the next
dictionary.) Will this dictionary be a
degenerate dictionary?

\medskip

\noindent\textbf{(C)} Write down the choice of the entering variable
for which the value of the objective function in the next dictionary
will be the highest? Will the value of the objective function depend
on the choice of the leaving variable as well (yes or no) ?

\medskip


\textbf{P2. (15 points)}  Consider the following feasible dictionary:

\[\begin{array}{r|c c c c c c c c}
x_{B,1} & b_1 & + a_{11} x_{N,1} & + \cdots & + \red{a_{1j}} x_{N,j} & \cdots & + a_{1n} x_{N,n}\\
\vdots & \vdots &  & \ddots & \vdots & \ddots  \\
x_{B,i} & b_i & +a_{i1} x_{N,1} & + \cdots & +\red{ a_{ij}} x_{N,j} & \cdots
                                                                & + a_{in} x_{N,n} \\
\vdots & & \ddots & \\ 
x_{B,m} & b_m & + a_{m1} x_{N,1} & + \cdots & +\red{ a_{mj} } x_{N,j} &
                                                                 \cdots  & + a_{mn} x_{N,n} \\
\hline
z & c_0 & + c_1 x_{N,1} & + \cdots & + \red{c_j} x_{N,j} & \cdots & + c_n x_{N,n}\\
\end{array}\]

\noindent \textbf{(A)} Suppose $x_{N,j}$ is chosen to enter and
$x_{B,i}$ is the corresponding leaving variable, then show that the
value of $x_{B,k}$ in the next dictionary after pivoting is given by
\[ b_k + a_{kj} \left(\frac{b_i}{ - a_{ij}}\right)  \,.\]

Also write down for each of the constants $b_k, a_{kj}, b_i, a_{ij}$
whether it is known that constant will be $> 0$, $< 0$, $\geq 0$,
$\leq 0$ or nothing may be said about its sign.

\medskip

\noindent \textbf{(B)} Show that if the leaving variable analysis is
correct then  the value of each basic variable $x_{B,k}$ in the
subsequent 
dictionary is  $\geq 0$. In other words: 
\[  b_k + a_{kj} \left(\frac{b_i}{ - a_{ij}}\right)  \geq 0 \,.\]

In other words, we conclude that starting from a feasible dictionary
and pivoting yields another feasible dictionary.

(\textbf{Hint:} Split two cases on the sign of $a_{kj}$. For one case
it will be trivially true. For the other, you have to appeal to the
leaving variable analysis as to why $x_{B,i}$ was the leaving variable and
not $x_{B,k}$).

\medskip


\noindent\textbf{(C)} Using the analysis above, prove that if $x_{B,k}$
and $x_{B,i}$ are both possible leaving variables ($i \not= k$) for $x_{N,j}$ entering,
then  the subsequent dictionary will be degenerate.  (\textbf{Hint:}
Assume that $x_{B,i}$ is chosen to leave. Show that even though $x_{B,k}$ did not leave the basis, its value in
the  next dictionary will be $0$).

\bigskip

\noindent\textbf{P3 (10 points)}  Provide examples of dictionaries
that satisfy the properties stated below. Try to construct
examples that are as small as possible. If no such
dictionaries can exist, briefly reason why.


\noindent\textbf{(A)} A degenerate dictionary that is also unbounded. 
Recall that an unbounded dictionary does not have a leaving variable for some
choice of an entering variable.

\medskip

\noindent\textbf{(B)} A degenerate dictionary $D$ which upon pivoting
yields another degenerate dictionary $D'$, but the objective value strictly
increases.

\medskip

\noindent\textbf{(C)} A non-degenerate dictionary $D$ which upon pivoting
yields another dictionary $D'$ but the value of the objective function
stays the same.

\medskip

\noindent\textbf{(D)} A dictionary that is feasible but upon pivoting
yields an infeasible dictionary.


\medskip

\noindent\textbf{(E)} A dictionary that does not have leaving variable
(is unbounded) for one choice of entering variable but has a leaving
variable for a different choice of an entering variable.

\bigskip

\noindent\textbf{P4 (15 points)}  Consider the polyhedron below:
\[ \begin{array}{ccccc}
 -x & + 2 y & + 2z & \leq & 2 \\
2 x & - y & + 2z & \leq & 2 \\
x & & & \geq & 0\\
& y& & \geq & 0 \\
& & z & \geq & 0 \\
\end{array}\]

\noindent\textbf{(A)} Compute all the vertices and for each vertex
write down if it is degenerate or non-degenerate.

\noindent\textbf{(B)} Consider the optimization problem:
\[ \begin{array}{rcccccc}
\max & 2x & + 3y & - 2z \\
\mathsf{s.t} &  -x & + 2 y & + 2z & \leq & 2 & \# Slack\ w_1\\
& 2 x & - y & + 2z & \leq & 2 & \# Slack\ w_2\\
& x & & & \geq & 0\\
& & y& & \geq & 0 \\
& & & z & \geq & 0 \\
\end{array}\]

Write down all the dictionaries corresponding to the degenerate
vertices. Use slack variables $w_1, w_2$ as indicated.

\noindent\textbf{(C)} Draw a graph whose nodes are the vertices
described in (A) with edges between adjacent vertices. 

\medskip

\noindent\textbf{(D, extra credit)} Given a polyhedron $P$, and for
each vertex of the polyhedron,  can you  write down an
objective function that is uniquely maximized only at that vertex and
no other vertex of $P$? 


\medskip

\noindent\textbf{(E, extra credit)} For any polyhedron $P$, the
polyhedral graph (also called its skeleton) is one where the nodes
form the vertices of the polyhedron, and the edges connect adjacent
vertices. Prove that this graph is strongly connected for any
$P$. I.e, given any two vertices $\mathbf{v}_1$ and $\mathbf{v_2}$
there is a path between them in this graph.

\medskip

\noindent\textbf{(F, extra credit)}  Prove that the graph in part (E)
for a d-dimensional polyhedron $P$ has the property that if any subset
of $d-1$ or fewer vertices in the graph are removed, it will still
remain strongly connected (This is called Balinski's theorem).

To illustrate this, draw the skeleton of a cube and remove any two
vertices from this skeleton. You will notice that there is a path in
this graph between any pair of remaining vertices.



\end{document}
